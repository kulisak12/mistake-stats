\documentclass[a4paper,11pt]{article}
\usepackage[utf8]{inputenc}
\usepackage[czech]{babel}
\usepackage[T1]{fontenc}
\usepackage{lmodern}

\usepackage{a4wide}
\usepackage{csquotes}
\usepackage{graphicx}
\usepackage{hyperref}

\setlength{\parskip}{0.5em}
\setlength{\parindent}{0em}
\hypersetup{colorlinks=true}

\title{\Huge Chyby v~orientačním běhu}
\author{David Klement}
\date{Září 2022}

\begin{document}

\maketitle
\thispagestyle{empty}
\pagebreak

Orientační běh kombinuje dvě dovednosti, rychlý běh a~čtení mapy. Závodníci se
snaží zaběhnout vytyčenou trať co nejrychleji. Časové ztráty mohou pramenit
z~obou těchto složek, jak z~pomalého běhu, tak z~chyb v~orientaci.

V~této práci se zaměřuji na ztráty způsobené chybami a~hledám vzory v~nich. Dále
se zaměřuji na historii svých vlastních chyb, ze které lze pozorovat průběžný
zisk zkušeností a~vylepšení dovedností vedoucí k~omezení chyb.

Zdrojové kódy jsou dostupné
v~\href{https://github.com/kulisak12/mistake-stats}{tomto repozitáři}.

\section*{Data}

Všechny oficiální závody konané v~České republice jsou dohledatelné v~systému
\href{https://oris.orientacnisporty.cz/?sport=1}{ORIS}. Tento systém poskytuje
\href{https://oris.orientacnisporty.cz/API}{API}, prostřednictvím kterého je možné
data ze závodů získat.

U~většiny závodů jsou dostupné mezičasy, tedy informace, v~jakém čase doběhl
jaký závodník na jakou kontrolu. Čas, který závodník strávil mezi dvěma
kontrolami, lze použít k~analýze chyb. Pokud je závodníkův čas významně vyšší
než čas ostatních závodníků, dá se předpokládat, že na postupu na danou kontrolu
udělal chybu.

Při stahování ukládám data na disk, aby se omezil počet požadavků na API.

\subsection*{Zpracování}

Stažené mezičasy je nutné trochu zpracovat. Někteří závodníci totiž závod
nemuseli dokončit, případně některou kontrolu mohli vynechat. Poté nejsou
dostupné informace o~času, který strávili postupem na danou a~následující
kontrolu. Pro jednoduchost tyto závodníky z~dat plně odstraňuji.

Do dat přidávám pouze ty závody, které dokončili alespoň 3~závodníci. Pro
2~závodníky totiž nelze rozlišit přirozený rozptyl časů od chyb.

\subsection*{Detekce chyb}

Není nijak definováno, co chyba je a~co není. Pokusil jsem se tedy vytvořit
systém, který dle mého rozumně odpovídá skutečnosti.

Jak jsem zmínil na začátku, čas závodníka je určen i~rychlostí běhu. Prvně tedy
musím rozlišit běžeckou ztrátu od ztráty způsobené chybou. Budu předpokládat, že
běžecká rychlost závodníka v~poměru s~ostatními je po celý závod stejná.
Tento předpoklad nutně nemusí platit, snadno se může stát, že někdo v~úvodu
závodu přecení své síly a~ke konci výrazně zpomalí. Bohužel, toto nejsem schopný
pouze na základě mezičasů odlišit od chyb.

Časy tedy normalizuji tak, aby celkový čas na celé trati byl pro každého
závodníka stejný. Tím se zbavím závislosti na rychlosti běhu.

Následně pro každého závodníka spočítám jeho ztrátu na postupu jako rozdíl jeho
času a~nejlepšího dosaženého času. Pokud ztráta převyšuje určitou mez, prohlásím
to za chybu.

Otázkou zůstává, jak tuto mez určit. Mez nebude vždy stejná, protože
v~orientačním běhu se závodní v~různých disciplínách lišících se délkou. Ztráta
30~vteřin by v~nejdelší disciplíně, na klasické trati, chybu nejspíše
neznamenala, zatímco ve sprintu se již jedná o~hrubou chybu. Dává tedy smysl
uvažovat ztrátu v~poměru k~délce závodu.

Rozhodl jsem se zavést dvě třídy chyb:
\begin{enumerate}
    \item Malé chyby, u~kterých ztráta na postupu převyšuje $1/50$~délky závodu.
          To by podle oficiálních směrných časů v~kategorii H21 znamenalo $14\,\rm s$
          na sprintu, $42\,\rm s$ na krátké trati a~$90\,\rm s$ na klasické trati.
    \item Velké chyby, u~kterých ztráta převyšuje $1/20$~délky závodu, což
          odpovídá ztrátě $36\,\rm s$, $105\,\rm s$, respektive $225\,\rm s$.
\end{enumerate}

\subsection*{Dataset}

Využívám data ze všech závodů do konce srpna~2022, kterých jsem se kdy účastnil.
Celkem se jedná o~83 závodů.

V~některých statistikách průměruji všechny závodníky dohromady, v~některých
uvažuji jen své výsledky.
\pagebreak

\section*{Hypotéza první kontroly}

První kontrola v~závodě mnohdy bývá nejzrádnější, neboť závodník ještě není plně
soustředěný. Osobně mám pocit, že na první kontrolu udělám chybu poměrně často.
Chci tedy zjistit, jestli tak tomu doopravdy je.

Nulová hypotéza je, že chyby nastávají stejně často na první i~na dalších
kontrolách. Budu zkoumat malé i~velké chyby, pro sebe i~pro všechny závodníky.
Použiji Bonferroniho korekci a~hladinu významnosti nastavím jako
$\alpha = 0.05 / 4 = 0.0125$.

Označme~$X$ Bernoulliovskou náhodnou veličinu odpovídající chybě na první
kontrole, $Y$~odpovídající chybě na dalších kontrolách. Testová statistika je
rozdíl jejich průměrů $\overline{X}_n - \overline{Y}_m$. Dle CLV má tato
statistika přibližně normální rozdělení, rozptyl odhadnu jako
$p (1-p) (1/n + 1/m)$, kde $p$ je průměr všech hodnot $X$ a~$Y$.

Pro malé chyby a~všechny závodníky vychází pro prvních pár kontrol následující
pravděpodobnosti chyb:

\begin{tabular}{c | c}
    Kontrola & Pravděpodobnost \\
    \hline
    1        & 0.07684         \\
    2        & 0.06961         \\
    3        & 0.05972         \\
    4        & 0.07417         \\
    5        & 0.06618         \\
\end{tabular}

Na první kontrole bylo 202~chyb z~2\,629, na ostatních 2\,901 z~46\,623. Při
použití testu výše vychází p-hodnota~0.00149, což by znamenalo, že bych měl
nulovou hypotézu zamítnout. To je zvláštní, z~pravděpodobností výše je vidět, že
jsou výsledky dost zašuměné.

Problém je v~tom, že data nejsou nezávislá. Je-li jedna z~kontrol na trati
těžká, nejspíše na ní udělá chybu velká část závodníků běžící daný závod. Test
tedy pozměním, veličina~$X$ bude označovat procento závodníků, kteří v~daném
závodě udělali chybu na první kontrole. $Y$~zavedu analogicky. Tím bych měl
zajistit nezávislost veličin.

Provedu t-test s~alternativní hypotézou, že $X$~je větší než~$Y$. Výsledky:

\begin{tabular}{r | l | l | l}
     & $\overline{X}_n$ & $\overline{Y}_m$ & p-value  \\
    \hline
    malé chyby, všichni závodníci  & 0.10648 & 0.07984 & 0.02354  \\
    velké chyby, všichni závodníci & 0.03532 & 0.01881 & 0.00407  \\
    malé chyby, pouze já           & 0.20000 & 0.08579 & 0.00046  \\
    velké chyby, pouze já          & 0.12000 & 0.02439 & 0.000001 \\
\end{tabular}

S~výjimkou prvního případu lze nulovou hypotézu vyvrátit. Zdá se tedy, že na
první kontrole opravdu nastávají chyby častěji. Já zejména bych si dle výsledků
měl na první kontrolu dávat obzvlášť pozor.

\section*{Hypotéza dvojchyb}

Podobně motivovaná je také hypotéza, že jsou chyby častější v~případě, kdy
závodník chyboval na předchozí kontrole. Taková chyba závodníka rozhodí a~mnohdy
vyústí v~další chybu.

Budu tedy hledat podmíněnou pravděpodobnost chyby za předpokladu, že závodník na
předchozí kontrole chyboval, a~porovnávat ji s~toutéž pravděpodobností za
předpokladu, že předchozí kontrolu našel \uv{čistě}.

Myslím, že nemá cenu tento test provádět pro velké chyby. Dvě velké chyby za
sebou se moc často nevidí. Hodnota významnosti tedy budiž $0.05 / 2 = 0.025$.

Nulová hypotéza opět bude, že jsou tyto pravděpodobnosti stejné. Poučen z~minula
volím náhodné veličiny tak, aby z~jednoho závodu vzešlo jedno měření.

\begin{tabular}{r | l | l | l}
    & chyba po chybě & chyba po čisté kontrole & p-value \\
   \hline
   všichni závodníci  & 0.08754 & 0.08616 & 0.45022 \\
   pouze já           & 0.11178 & 0.12385 & 0.66285 \\
\end{tabular}

Hypotéza dvojchyb evidentně neplatí.

\section*{Historie chyb}

V~poslední řadě se podívám na historii svých chyb. Zajímalo by mě, jestli dělám
chyb méně než dříve. Zde už poměrně záleží na tom, jaké závody v~datech
používám. Jednak jsem za ty roky prošel několika kategoriemi, ve kterých
náročnost tratí roste. Mimo to jezdím častěji na republikové závody, které
taktéž bývají těžší.

Pro každý závod spočítám, na kolika procentech kontrol jsem udělal chybu. Tuto
hodnotu vynesu na osu~$y$, za osu~$x$ zvolím pořadové číslo závodu, číslováno
chronologicky. Nakonec spočítám lineární regresi a~tím zjistím trend chybovosti.

Uvažoval jsem pouze hranici malých chyb. Od mých začátků počet mých chyb klesl
zhruba na polovinu. Výsledek je vidět v~následujícím grafu:

\begin{center}
    \includegraphics[width=0.8\textwidth]{history-probs.png}
\end{center}

\end{document}
